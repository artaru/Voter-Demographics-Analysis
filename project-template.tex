% Options for packages loaded elsewhere
\PassOptionsToPackage{unicode}{hyperref}
\PassOptionsToPackage{hyphens}{url}
%
\documentclass[
  ignorenonframetext,
]{beamer}
\usepackage{pgfpages}
\setbeamertemplate{caption}[numbered]
\setbeamertemplate{caption label separator}{: }
\setbeamercolor{caption name}{fg=normal text.fg}
\beamertemplatenavigationsymbolsempty
% Prevent slide breaks in the middle of a paragraph
\widowpenalties 1 10000
\raggedbottom
\setbeamertemplate{part page}{
  \centering
  \begin{beamercolorbox}[sep=16pt,center]{part title}
    \usebeamerfont{part title}\insertpart\par
  \end{beamercolorbox}
}
\setbeamertemplate{section page}{
  \centering
  \begin{beamercolorbox}[sep=12pt,center]{part title}
    \usebeamerfont{section title}\insertsection\par
  \end{beamercolorbox}
}
\setbeamertemplate{subsection page}{
  \centering
  \begin{beamercolorbox}[sep=8pt,center]{part title}
    \usebeamerfont{subsection title}\insertsubsection\par
  \end{beamercolorbox}
}
\AtBeginPart{
  \frame{\partpage}
}
\AtBeginSection{
  \ifbibliography
  \else
    \frame{\sectionpage}
  \fi
}
\AtBeginSubsection{
  \frame{\subsectionpage}
}
\usepackage{lmodern}
\usepackage{amssymb,amsmath}
\usepackage{ifxetex,ifluatex}
\ifnum 0\ifxetex 1\fi\ifluatex 1\fi=0 % if pdftex
  \usepackage[T1]{fontenc}
  \usepackage[utf8]{inputenc}
  \usepackage{textcomp} % provide euro and other symbols
\else % if luatex or xetex
  \usepackage{unicode-math}
  \defaultfontfeatures{Scale=MatchLowercase}
  \defaultfontfeatures[\rmfamily]{Ligatures=TeX,Scale=1}
\fi
\usetheme[]{Pittsburgh}
\usecolortheme{orchid}
\usefonttheme{structurebold}
% Use upquote if available, for straight quotes in verbatim environments
\IfFileExists{upquote.sty}{\usepackage{upquote}}{}
\IfFileExists{microtype.sty}{% use microtype if available
  \usepackage[]{microtype}
  \UseMicrotypeSet[protrusion]{basicmath} % disable protrusion for tt fonts
}{}
\makeatletter
\@ifundefined{KOMAClassName}{% if non-KOMA class
  \IfFileExists{parskip.sty}{%
    \usepackage{parskip}
  }{% else
    \setlength{\parindent}{0pt}
    \setlength{\parskip}{6pt plus 2pt minus 1pt}}
}{% if KOMA class
  \KOMAoptions{parskip=half}}
\makeatother
\usepackage{xcolor}
\IfFileExists{xurl.sty}{\usepackage{xurl}}{} % add URL line breaks if available
\IfFileExists{bookmark.sty}{\usepackage{bookmark}}{\usepackage{hyperref}}
\hypersetup{
  pdftitle={A stastical analysis of voter demographics for the Liberal Party of Canada},
  pdfauthor={Artem Arutyunov},
  hidelinks,
  pdfcreator={LaTeX via pandoc}}
\urlstyle{same} % disable monospaced font for URLs
\newif\ifbibliography
\usepackage{longtable,booktabs}
\usepackage{caption}
% Make caption package work with longtable
\makeatletter
\def\fnum@table{\tablename~\thetable}
\makeatother
\usepackage{graphicx,grffile}
\makeatletter
\def\maxwidth{\ifdim\Gin@nat@width>\linewidth\linewidth\else\Gin@nat@width\fi}
\def\maxheight{\ifdim\Gin@nat@height>\textheight\textheight\else\Gin@nat@height\fi}
\makeatother
% Scale images if necessary, so that they will not overflow the page
% margins by default, and it is still possible to overwrite the defaults
% using explicit options in \includegraphics[width, height, ...]{}
\setkeys{Gin}{width=\maxwidth,height=\maxheight,keepaspectratio}
% Set default figure placement to htbp
\makeatletter
\def\fps@figure{htbp}
\makeatother
\setlength{\emergencystretch}{3em} % prevent overfull lines
\providecommand{\tightlist}{%
  \setlength{\itemsep}{0pt}\setlength{\parskip}{0pt}}
\setcounter{secnumdepth}{-\maxdimen} % remove section numbering
\def\begincols{\begin{columns}}
\def\begincol{\begin{column}}
\def\endcol{\end{column}}
\def\endcols{\end{columns}}

\title{A stastical analysis of voter demographics for the Liberal Party of
Canada}
\author{Artem Arutyunov}
\date{December 7, 2022}

\begin{document}
\frame{\titlepage}

\begin{frame}{Introduction}
\protect\hypertarget{introduction}{}

A crucial set of information for any political strategist consists of
voter demographics. A political party must know their current voter
demographics to plan an effective campaign which will maximize voter
behaviour in their favour.

In this presentation, we conduct a statistical analysis of voter
demographics for the purpose of advising the \emph{Liberal Party of
Canada}. We investigate the following research questions:

\begin{enumerate}
\item
  \textbf{Are people who selected Liberal as their first choice to vote
  equally as likely to support more or less refugees?}
\item
  \textbf{Does gender influence the median rating of Justin Trudeau?}
\item
  \textbf{Can we predict voter behaviour based on traits such as age,
  gender, and marital status?}
\end{enumerate}

\end{frame}

\begin{frame}[fragile]{Data Summary}
\protect\hypertarget{data-summary}{}

We used the data collected from the Canadian Election Study in 2019
(\texttt{ces19}), which consists of \(37,822\) responses from online and
phone surveys.

\textbf{Variables used:}

\begin{enumerate}
\item
  Vote choice (first choice to vote for): will be used to filter out
  people who are not voting for the Liberal Party, and to determine
  voting preference.
\item
  Refugee (should Canada admit more refugees?): will be used to classify
  the opinion of the voters. If their answer is more refugees, then they
  are supporting the admission, if it is fewer refugees, then they are
  against the admission. Answers such as ``same number of refugees'' or
  ``don't know/prefer not to'', will be ignored because this opinion can
  not be classified as strictly positive or negative.
\end{enumerate}

\end{frame}

\begin{frame}{Research Question I}
\protect\hypertarget{research-question-i}{}

Political parties should pay attention to the opinion of voters on
important discussions that are present in Canadian society. One such
issue regards immigrants and refugees.

For this question, we examine whether Liberal voters are equally as
likely to support or oppose policies regarding admitting more refugees
into Canada.

\textbf{Null hypothesis} (\(H_0\)): The proportion of voters whose first
choice is the Liberal party who support more refugees is 50\%.
\[H_0: p_\text{more} = 0.5\] \textbf{Alternative hypothesis} (\(H_1\)):
The proportion of voters whose first choice is the Liberal party who
support more refugees is not 50\%. \[H_1: p_\text{more} \neq 0.5\] where
\(p_\text{more}\) refers to the proportion of voters.

\end{frame}

\begin{frame}{Statistical Methods}
\protect\hypertarget{statistical-methods}{}

To test the null hypothesis, we conducted a series of simulations in R.

\begin{itemize}
\tightlist
\item
  To find the test statistic, the proportion of Liberal voters
  (i.e.~first choice is Liberal) who support the admission of refugees
  was calculated (\(\hat{p}_\text{more}=0.4242\)).
\item
  To get a sampling distribution, we ran a test where we randomly sample
  a position (``support'' or ``oppose'') with uniform probability for
  each voter. We repeated this test \(10,000\) times. For each test, the
  proportion of positive answers was calculated.
\item
  The p-value was found by calculating the proportion of values in the
  estimated sampling distribution that are more or as extreme as the
  test statistic.
\end{itemize}

\end{frame}

\begin{frame}{Results \& Conclusions}
\protect\hypertarget{results-conclusions}{}

\begincols
\begincol{0.6\textwidth}
\begin{itemize}
\item From our randomization test, we computed a $p$-value of $0$.
\item Since $P=0<0.01$, the probability of observing a value \emph{at least as extreme} as the test statistic ($\hat{p}_\text{more}=0.4242$) is \emph{extremely} unlikely. We have \textit{very strong} evidence against the null hypothesis.
\end{itemize}
\vspace{1em}

Therefore, we can conclude that the proportion of liberal party voters
who support the policy is \textbf{not} \(50\%\). In other words, this
means that it is not equally likely for Liberal voters to take a random
stance on the refugee issue.

\endcol
\begincol{0.4\textwidth}
\fontsize{6}{7.2}\selectfont

\includegraphics{project-template_files/figure-beamer/unnamed-chunk-5-1.pdf}
\textbf{Figure 1}: the distribution of the original data. As we can see,
the distribution of the raw data is consistent with our conclusions from
the randomization test (that \(p_\text{more}\neq 0.5\)).

\endcol
\endcols

\end{frame}

\begin{frame}[fragile,t]{Research Question II}
\protect\hypertarget{research-question-ii}{}

\begincols
\begincol{0.7\textwidth}

For this question, we examine whether gender influences the median
rating of party leader, Justin Trudeau? \vspace{1em}

\textbf{Null hypothesis} (\(H_0\)): There is no difference between the
median rating of Justin Trudeau (\texttt{lead\_rating\_23}) between men
and women:
\[H_0: \text{median}_\text{men}-\text{median}_\text{women} = 0.\]

\textbf{Alternative hypothesis} (\(H_1\)): There is a difference between
the median rating of Justin Trudeau (\texttt{lead\_rating\_23}) between
men and women:
\[H_1: \text{median}_\text{men}-\text{median}_\text{women} \neq 0.\]
\endcol \begincol{0.3\textwidth} \fontsize{6}{7.2}\selectfont

\includegraphics[width=0.8\textwidth,height=\textheight]{justin.png}
\textbf{Figure 2}: Justin Trudeau: Benevolent Party Leader and Prime
Minister.

\vspace{2em}

\includegraphics[width=0.8\textwidth,height=\textheight]{scheer.png}
\textbf{Figure 3}: Opposing party leader, Andrew Scheer, drinks Neilson
Dairy 2\% skim milk.

\endcol
\endcols

\end{frame}

\begin{frame}{Statistical Methods}
\protect\hypertarget{statistical-methods-1}{}

To test the null hypothesis, we conducted a series of simulations in R
(similar to the method used in part A).

\begin{itemize}
\tightlist
\item
  The test statistic is the difference between the median rating among
  men and the median rating among women:
  \[\hat{\text{median}}_\text{men}-\hat{\text{median}}_\text{women}=8.\]
\item
  To get a sampling distribution, we ran a test where we randomly
  shuffle the gender labels for each observation with uniform
  probability. We repeated this test \(10,000\) times. For each test,
  the difference between the median ratings was calculated.
\end{itemize}

\end{frame}

\begin{frame}[t]{Results \& Conclusions}
\protect\hypertarget{results-conclusions-1}{}

\begincols
\begincol{0.4\textwidth}
\fontsize{6}{7.2}\selectfont

\includegraphics{project-template_files/figure-beamer/unnamed-chunk-10-1.pdf}
\textbf{Figure 4}: the distribution of ratings in the original data. The
distribution of the raw data is consistent with our conclusions from the
randomization test. \vspace{1em} \endcol \begincol{0.6\textwidth}

\begin{itemize}
\item From our randomization test, we computed a $p$-value of $0$.
\item Since $P=0<0.01$, we have \textit{very strong} evidence against the null hypothesis.
\item Therefore, we can conclude there \textbf{is} a difference of median rating of Justin Trudeau between men and women.
\end{itemize}
\endcol
\endcols

These results show that when campaigning and advertising, we should
consider the gender of potential voters to better target their
expectations of party leaders. If our data is representative of the
real-world, according to Figure 4, this may suggest that on the whole,
women rate Trudeau higher than men.

\end{frame}

\begin{frame}{Research Question III}
\protect\hypertarget{research-question-iii}{}

Broadly speaking, are there certain traits that voters have that we can
use to predict who they will vote for? Can age, gender, and marital
status be used to predict a voter's choice? If we add more than those 3,
will we have a clearer picture of predicted voter choice?

For this question, we will focus our attention on predicting whether a
voter will vote for the top two parties of Canada: the Liberals or
Conservatives.

\end{frame}

\begin{frame}[fragile]{Statistical Methods}
\protect\hypertarget{statistical-methods-2}{}

We trained multiple classification models to predict voter choice
(\texttt{votechoice}) based on the following variables:

\begin{itemize}
\tightlist
\item
  \texttt{age}, \texttt{gender}, \texttt{province},
  \texttt{citizenship}, \texttt{bornin\_canada}
\item
  \texttt{sexuality}, \texttt{religion}, \texttt{marital},
  \texttt{education}, \texttt{union}, \texttt{children}
\item
  \texttt{groups\_therm\_1} to \texttt{groups\_therm\_5}
\item
  \texttt{interest\_gen\_1}, \texttt{interest\_elxn\_1}
\item
  opinions on spending: \texttt{spend\_educ}, \texttt{spend\_env},
  etc\ldots{}
\item
  opinions on the economy: \texttt{econ\_retro}, etc\ldots{}
\item
  opinions on immigrants/refugees: \texttt{imm}, \texttt{refugees}
\item
  opinions on the government: \texttt{govt\_confusing},
  \texttt{govt\_say}, and \texttt{lib\_promises}.
\end{itemize}

\begin{block}{Why so much data?}

To accurately train a classification model, we need sufficient
predictors to distinguish voters of different parties.

\end{block}

\end{frame}

\begin{frame}{Statistical Methods (cont'd.)}
\protect\hypertarget{statistical-methods-contd.}{}

\begincols
\begincol{0.7\textwidth}
\fontsize{9}{12.2}\selectfont

The classification models were compared with one another on the basis of
accuracy to find the most accurate model. We split the data into a
training and test set (80/20), and then used them to train and evaluate
the following models:

\begin{itemize}
\item
  \textbf{Classification tree}: by asking TRUE/FALSE questions about the
  predictors, we can narrow down the set of possible outputs, and if we
  keep doing this, we will converge to a single answer. Formally, each
  `question' is called a binary split.
\item
  \textbf{Neural Network}: a simplified model of the brain; mimics its
  structure with the hopes of finding relationships in the predictors
  (similar to that of a human).
\item
  \textbf{Support Vector Machine}: an approach for ``dividing'' our
  predictors into regions, one for each class of the output.
\end{itemize}

\endcol
\begincol{0.3\textwidth}
\fontsize{6}{7.2}\selectfont

\includegraphics[width=0.8\textwidth,height=\textheight]{svm.png}
\textbf{Figure 4}: Visualisation of the support vector machine.
\vspace{1em}

\includegraphics[width=0.8\textwidth,height=\textheight]{nn.png}
\textbf{Figure 5}: A simple neural network. \endcol \endcols

\end{frame}

\begin{frame}{Results}
\protect\hypertarget{results}{}

\begincols
\begincol{0.6\textwidth}

The table summarises the accuracy between the classification tree,
neural network, and support vector machine. From the table, we see that
the support vector machine marginally outperforms neural network, and
both of them outperform the classification tree.

\begin{longtable}[]{@{}ll@{}}
\toprule
Model & Accuracy\tabularnewline
\midrule
\endhead
Classification Tree & 0.8485096\tabularnewline
Neural Network & 0.8730094\tabularnewline
\textbf{Support Vector Machine} & 0.8961002\tabularnewline
\bottomrule
\end{longtable}

\endcol
\begincol{0.4\textwidth}
\fontsize{6}{7.2}\selectfont

\includegraphics{project-template_files/figure-beamer/unnamed-chunk-26-1.pdf}
\textbf{Figure 6}: Confusion matrix for the classification tree.

\includegraphics{project-template_files/figure-beamer/unnamed-chunk-27-1.pdf}
\textbf{Figure 7}: Confusion matrix for the support vector machine.

\endcol
\endcols

\end{frame}

\begin{frame}{Conclusion}
\protect\hypertarget{conclusion}{}

Our results showed that the support vector machine outperforms the other
two models; however, all three models we tested scored fairly high in
terms of accuracy. This shows that we can predict voter choice based on
traits such as age, gender, and marital status.

\begin{block}{Limitations}

\begin{itemize}
\item
  For neural networks to perform well, and not overfit, we need enough
  data. Insufficient data (number of observations) was the biggest
  limitation. Hence, we could further improve the accuracy with more
  observations.
\item
  The classification tree is sensitive to noise (a small change in the
  data can cause a large change in the tree structure), making it an
  unstable classifier.
\item
  Similarly, the support vector machine does not perform well when there
  is noise, and complex structure in the data (unpredictable
  interdependence).
\end{itemize}

\end{block}

\end{frame}

\begin{frame}{Summary of Report \& Future Insights}
\protect\hypertarget{summary-of-report-future-insights}{}

\begin{enumerate}
\item
  Since the proportion of liberal party voters that support refugees is
  not 50\%, the party should investigate to which direction the voters
  lean, to understand the stance that their voters are taking.
\item
  The medians are different with the women in the data set favoring
  Justin Trudeau more than men do. The party needs to investigate why,
  and work towards decreasing the divide between sentiment among the
  genders.
\item
  It is possible to predict which party a person will vote for based off
  of their traits, and the party needs to make sure they keep those that
  are already voting for them; however, the party should also reach out
  to the voters that could be swung over between parties.
\end{enumerate}

\end{frame}

\begin{frame}{References and Acknowledgements}
\protect\hypertarget{references-and-acknowledgements}{}

\fontsize{8}{10}\selectfont

\begin{itemize}
\item
  {[}1{]}: Wikimedia Commons contributors, ``\url{File:Neural} network
  example.svg,'' Wikimedia Commons, the free media repository,
  \url{https://commons.wikimedia.org/w/index.php?title=File:Neural_network_example.svg\&oldid=459217583}
  (accessed December 6, 2020).
\item
  {[}2{]}: Wikimedia Commons contributors, ``\url{File:Kernel}
  Machine.svg,'' Wikimedia Commons, the free media repository,
  \url{https://commons.wikimedia.org/w/index.php?title=File:Kernel_Machine.svg\&oldid=467887503}
  (accessed December 6, 2020).
\item
  {[}3{]}: Selley, Chris. ``Chris Selley: Could Scheer's Devotion to Big
  Dairy Finally Break the Supply Management Consensus?''. National Post,
  July 19, 2019.
  \url{https://nationalpost.com/opinion/chris-selley-could-scheers-devotion-to-big-dairy-finally-break-the-supply-management-consensus}.
\end{itemize}

\end{frame}

\end{document}
